% $Header: /Users/joseph/Documents/LaTeX/beamer/solutions/generic-talks/generic-ornate-15min-45min.de.tex,v 90e850259b8b 2007/01/28 20:48:30 tantau $

\documentclass{beamer}
\long\def\/*#1*/{}



\mode<presentation>
{
  \usetheme{Rochester}
}
\definecolor{MyBlue}{RGB}{0, 130 210}

\usepackage[german]{babel}
% oder was auch immer

\usepackage[latin1]{inputenc}
% oder was auch immer

\usepackage{times}
\usepackage[T1]{fontenc}
% Oder was auch immer. Zu beachten ist, das Font und Encoding passen
% m�ssen. Falls T1 nicht funktioniert, kann man versuchen, die Zeile
% mit fontenc zu l�schen.


\title[University of Amsterdam] % (optional, nur bei langen Titeln n�tig)
{TERESA : Telepresence Reinforcement Learning Social Agent.}

\subtitle
{Learning Social Skills} % (optional)

\author[Kyriacos Shiarlis] % (optional, nur bei vielen Autoren)
{Kyriacos Shiarlis}
% - Der \inst{?} Befehl sollte nur verwendet werden, wenn die Autoren
%   unterschiedlichen Instituten angeh�ren.

\institute[University of Amsterdam] % (optional, aber oft n�tig)
{University of Amsterdam}
% - Der \inst{?} Befehl sollte nur verwendet werden, wenn die Autoren
%   unterschiedlichen Instituten angeh�ren.
% - Keep it simple, niemand interessiert sich f�r die genau Adresse.

\date[] % (optional)
{16/06/2014}

\AtBeginSection[]
{
  \begin{frame}<beamer>{}
    \tableofcontents[currentsection]
  \end{frame}
}


% Falls Aufz�hlungen immer schrittweise gezeigt werden sollen, kann
% folgendes Kommando benutzt werden:

%\beamerdefaultoverlayspecification{<+->}



%%%%%%%%%%%%%%%%%%%%%%%%%%%%% REMEMBER IN PRESENTATIONS %%%%%%%%%%%%%%%%%%%%%%%%%%%%%%%%%%%%%%%%
% CONNECTIONS BETWEEN SLIDES
% NEVER EVER ME HUMBLE
% ALWAYS CONNECT BACK INTO THE APPLICATION
% DONT EVER SAY EEEEEE 
%%%%%%%%%%%%%%%%%%%%%%%%%%%%%%%%%%%%%%%%%%%%%%%%%%%%%%%%%%%%%%%%%%%%%%%%%%%%%%%%%%%%%%%%%%%%%%%%%%

\begin{document}



\begin{frame}
  \titlepage
\end{frame}

\begin{frame}{}
  \tableofcontents[]
  % Die Option [pausesections] k�nnte n�tzlich sein.
\end{frame}

%% GENERAL PROJECT DESCRIPTION - - - - - - - - - - - - - - - - - - - - - - - - - - - - - - - - - - - - - - - - - - - - - - 
\section{The Project}
\begin{frame}{Telepresence - What?}
	\begin{center}
	 Remotely controlled robots that allow the user to interact with an environment, without being physically present.
	 
	\includegraphics[scale = 0.4]{SNOW.jpg}		
	\end{center}
\end{frame}
% PROJECT MOTIVATION
\begin{frame}{Telepresence - Why?}
	Telepresence allows greater \textbf{control} and \textbf{interaction} for the remote user.\\
	\vspace{3mm}
	\uncover<2->{The user also \textbf{feels} and \textbf{appears} more present}\\
	\vspace{3mm}
	\uncover<3->{\underline{Applications include:}
			\begin{itemize}
				\item \textbf{Assistive technologies:} Remote visits to elderly, disabled, or hospitalised individuals.	
				\item \textbf{Industrial:} Remote inspections, conferences, visits.
				\item \textbf{Academic:} Conferences, supervisions. 
			\end{itemize}}
	\setbeamercolor{postit}{fg=white,bg=MyBlue}
	\uncover<4->{\begin{beamercolorbox}[sep=0.5em,wd=10cm]{postit}
					TERESA concentrates on deployment in elderly homes.
				\end{beamercolorbox}}
\end{frame}

% WHATS MISSING
\begin{frame}{Limitations}
	\begin{columns}
		\begin{column}[t]{5cm}
			\begin{figure}
				\includegraphics[width = 35mm, height =50mm]{ROBOT.jpg}
			\end{figure}
		\end{column}
		\begin{column}[t]{6cm}
			\begin{itemize}
				\item \textbf{Control} of the device can be hard.\\ % TODO EXPLAIN THE COGNITIVE OVERLOAD. What is it, why is it better to relieve it 
				\vspace{3mm}
				\item Interaction is not as \textbf{natural} as a result.\\
				\vspace{3mm}
				\item Device only allows \textbf{audiovisual} interaction.\\
			\end{itemize}
		\end{column}
	\end{columns}
\end{frame}

% PROJECT AIMS
\begin{frame}{Project Aims}
	\begin{block}{Practical}
		\begin{itemize}
			\item Remove the cognitive load of control.
			\item Appear socially integrated. 
		\end{itemize}
	\end{block}
	\begin{block}<2->{Scientific}
		\begin{itemize}
			\item To what extent socially acceptable behaviour can be Learned.
			\item What sort of implicit feedback is needed to achieve this.
		\end{itemize}
	\end{block}
\end{frame}


% EXAMPLE
\begin{frame}{Example}
	\begin{block}{Questions}
		How should a robot approach a group of people? What is the correct distance to stop?
	\end{block}
	\uncover<2->{$\Rightarrow$ Hard Coding Social Norms is very complex.}
	\begin{block}<3->{Our approach}
		Experiment $\rightarrow$ Data $\rightarrow$ Offline Learning $\rightarrow$ Semi-autonomous behaviour.	
	\end{block}
		\uncover<4>{$\Rightarrow$ More fluent and natural local-remote user interaction.}
\end{frame} 

% CONGITIVE ARCHITECTURE
\begin{frame}{Cognitive Architecture} % TODO: Explain this better. More detail about what the partners do. Social part. Sensing part. Makes our purpose more clear as well
	\begin{columns}
	\hspace{-30mm}
		\begin{column}[t]{3.2cm}
			
			Feedback from:
			\begin{itemize}
				\item Facial analysis.
				\vspace{4mm}
				\item Conversation flow/tone.
				\vspace{4mm}
				\item Body poses.
			\end{itemize}
		\end{column}
		\hspace{-25mm}
		\begin{column}[t]{3cm}
			
				\includegraphics[width = 67mm,height = 64mm]{framework.pdf}
			
		\end{column}
	\end{columns}
\end{frame}

%% Learning Social Skills - - - - - - - - - - - - - - - - - - - - - - - - - - - - - - - - - - - - - - - - - - - - - - - - - - - - - - - - -
%Questions
\section{Learning Social Skills}
\begin{frame}{Learning Social Skills}
	\uncover<1->{How can emotional/implicit feedback from the robot's environment improve its behaviour?}\\
	\begin{block}<2->{Example}
		Robot comes \textbf{dangerously} close and at high velocity - Person \textbf{frowns} - After learning the robot \textbf{avoids} action in similar situations. 	
	\end{block}
	\uncover<3->{Does that perform better than hand-coding social behaviour?}		
\end{frame}

% AIMS
\begin{frame}{Learning Social Skills - Aims} % THREE FORMS OF FEEDBACK. explain. better. make clear why there are 3 of them and what they serve
	\begin{center}
		\begin{columns}
			\begin{column}[t]{3cm}<1->
				\underline{\textbf{Extract}}
			\end{column}
			\begin{column}[t]{2cm}<2->
				\underline{\textbf{Integrate}}
			\end{column}
			\begin{column}[t]{1cm}<3->
				\underline{\textbf{Plan}}
			\end{column}
		\end{columns}
	\end{center}
	\includegraphics[scale = 0.29]{MODULES.png}
\end{frame}

%Extract
\begin{frame}{Challenges}
	\begin{block}{Extraction}	
		Extracting reward from the environment is an exercise in implicit feedback.
		\begin{itemize}
			\item Semi-Supervised Learning : Implicit emotional state $\Rightarrow$ Reward.			
			\item Inverse Reinforcement Learning: Expert trajectories $\Rightarrow$ Reward
		\end{itemize}
	\end{block}
	\begin{center}
		\includegraphics[scale = 0.4]{EXTRACT.png}
	\end{center}
\end{frame}
\begin{frame}{Challenges}
	\begin{block}{Integration}
		Integration of cost functions should be done intelligently.
		\begin{itemize}
			\item Could be based on individual function confidence.
			\item Bayesian Approach.
		\end{itemize}
	\end{block}
	\begin{center}
		\includegraphics[scale = 0.4]{INT.png}
	\end{center}
\end{frame}
\begin{frame}{Challenges}
	\begin{block}{Planning}
		\begin{itemize}
			\item<1-> UvA is responsible for planning body pose policies.
			\item<2-> What are the priorities in social occasions?
			\item<4-> Collaborating with UPO on Navigation.
			\item<5-> How will the two be regulated?
		\end{itemize}
	\end{block}
	\begin{center}
		\includegraphics[scale = 0.35]{PLAN.png}
	\end{center}
\end{frame}

%% CURRENT RESEARCH INTERESTS-----------------------------------------------------------------------------------------------------------------------------------------------
\section{Current Research Intrests} % TODO: make an introduction regarding apprenticeship learning in general. Show definitions and limitations of each LfD - IRL. This makes the aim and assumptions of IRL better as well
% IRL
\begin{frame}{Inverse Reinforcement Learning}{Definition}  % Perhaps different cost functions for each scenario
	\begin{block}<2->{Given:} 
		\begin{enumerate}  
			\item Measurements of an agent's behaviour over time, in a variety of circumstances
			\item Sensory inputs to the agent.
			\item A model of the Environment.
		\end{enumerate}
	\end{block}
	\begin{block}<3->{Determine:} The reward function $R(s,a)$ being optimised. 
	\end{block}	
\end{frame}
\begin{frame}{Inverse Reinforcement Learning}
	\begin{itemize}
		\item An \textbf{apprentice} observes a state action trajectory $[(s_1,a_1), (s_2,a_2), ... ,  (s_T,a_T)]$  from an \textbf{expert}.
		\item $MDP_E = <S,A,T,\gamma,R>$ - R is hidden from the apprentice.
		\item Usually $R =w^T\phi(s,a)$ 
		\item So the IRL algorithm takes as input the trajectory and outputs the feature weights \pmb{w}.
		\item These are used by the apprentice to mimic and generalise the expert's preferences.
	\end{itemize}
\end{frame}
\begin{frame} {Inverse Reinforcement Learning}
Algorithms work by choosing weights to match certain trajectory statistics e.g:\\
\textbf{Feature Expectation} : $\Phi_E = \frac{1}{m}\sum_{m=0}^M\sum_{t=0}^T \phi(s_t,a_t)$\\
\textbf{Likelihood} : $P(s_{1-T},a_{1-T}|\pmb{w})$
	\begin{block}{Problems}<2->
		\begin{itemize}
			\item Many Reward functions will cause the observed behaviour. Additional constraints are many times used.
			\item Each iteration usually requires solving the MDP.
		\end{itemize}
	\end{block}
\end{frame}
% SURVEY
\begin{frame}{Many Approaches}
	\begin{block} {Max margin + Projection}
		Ng and Abbeel (2004) successfully applied their algorithms on simulated car driving.
	\end{block}
	\begin{block} {Max Entropy IRL}
		Ziebart et al (2010) added extra disambiguating constraints and applied to route prediction.
	\end{block}
	\begin{block}{Maximum Margin Planning}
		Ratliff et al (2006) Posed the problem as a Structured Classification. Again applied to route prediction.
	\end{block}
	Many more....But.
\end{frame}
% PARTIAL OBSERVABILITY
\begin{frame}{Partial Observability in IRL}
	\begin{block}{Observation 1:} All literature assumes the expert and apprentice have the same observational capabilities. \end{block} % explain this better
	\begin{block}{Observation 2:}<2-> No principled reason why IRL is better than imitation. \end{block}
	\begin{block}{Motivation}<3-> Partial observability is possible the case in TERESA. e.g: 
		\begin{itemize}
			\item The Pilot-Expert only senses through a camera.
			\item The Robot has 360 degree laser range finding capabilities.
		\end{itemize}
	\end{block}
	\setbeamercolor{postit}{fg=white,bg=MyBlue}	
	\uncover<4->{\begin{beamercolorbox}[sep=0.5em,wd=10cm]{postit}
		=> What are the implications of observability mismatch in IRL?
	\end{beamercolorbox}}
\end{frame}


%%% GIVE GENERAL EXAMPLE WITH THE PROJECT. Talk about different accuracies and mismatches in observability. THEN GO TO TIGER. 
%%% GOING TO TIGER IMMEDIATELLY CONFUSES PEOPLE. COUPLE EXTREME WITH TERESA SCENARIO. Identify the differences in toy and real. Seem like you know that toy is not the answer.


\begin{frame}{Extreme No 1}{Tiger Problem}
	\begin{center}
		Apprentice $\rightarrow$ partial observability \textbf{|} Expert $\rightarrow$ full observability\\
		\includegraphics[scale = 0.35]{Tiger1.png}
	\end{center}
\end{frame}


\begin{frame}{Extreme No 1}
	Apprentice $\rightarrow$ partial observability \textbf{|} Expert $\rightarrow$ full observability\\
	\vspace{-1mm}
	\begin{center}
		\includegraphics[scale = 0.45]{Diagram.png}
	\end{center}
	\vspace{-4mm}
		\begin{itemize}
			\uncover<1->{\item Apprentice receives belief-action trajectories $[(b(s)_1,a_1),(b(s)_2,a_2),...,(b(s)_T,a_T)]$}
			\uncover<2->{\item No information about what to do in uncertain belief states.}
			\uncover<3->{\item Less information about what the expert is trying to do!}
		\end{itemize}
	\setbeamercolor{postit}{fg=white,bg=MyBlue}	
\end{frame}

\begin{frame}{Extreme No 1}{Possible Solutions}
	\begin{block}<2->{What is the expert trying to do?}Perform forward-backward procedure on beliefs. This will push our samples to the extremes of the simplex.\end{block}
	\begin{block}<3->{What do we do when uncertain?}
		Assume a dual controller for the Apprentice.
			\begin{itemize}
				\item The information gathering part of the Reward function is given.
				\item The control part of the Reward function is learned from the expert trajectories.
			\end{itemize}
	\end{block}
\end{frame}


\begin{frame}{Extreme No 2}{Tiger Problem}
	Apprentice $\rightarrow$ full observability \textbf{|} Expert $\rightarrow$ partial observability\\
	\begin{center}
		\includegraphics[scale = 0.35]{Tiger2.png}
	\end{center}
\end{frame}

\begin{frame}{Extreme No 2}
	Apprentice $\rightarrow$ full observability \textbf{|} Expert $\rightarrow$ partial observability\\
	\vspace{-1mm}
	\begin{center}
		\includegraphics[scale = 0.42]{Diagram2.png}
	\end{center}
	\vspace{-5mm}
		\begin{itemize}
			\uncover<1->{\item Apprentice receives state-belief(belief)-action trajectories $[(s_1,b_A(b_E(s))_1,a_1),(s_1,b_A(b_E(s))_2,a_2),...,(s,b_A(b_E(s))_T,a_T)]$}
			\uncover<2->{\item We don't want to know what to do in uncertain belief states.}
			\uncover<3->{\item Less information about what the expert is trying to do!}
		\end{itemize}
	\setbeamercolor{postit}{fg=white,bg=MyBlue}	
\end{frame}

\begin{frame}{Extreme No 2}{Possible Solutions}
	\begin{block}<2->{What is the expert trying to do?}
		Perform forward-backward procedure on beliefs of beliefs. Again, this will push our samples to the extremes of the simplex.\end{block}
	\begin{block}<3->{How do we ignore uncertain states?}
		Assume the expert is using a dual-controller.
			\begin{itemize}
				\item The information gathering part of the Reward function is given.
				\item The control part of the Reward function is learned from the expert trajectories.
			\end{itemize}
	\end{block}
\end{frame}

\begin{frame}{Partial Observability IRL}{Conclusions}
	\begin{centering}
	\setbeamercolor{postit}{fg=white,bg=MyBlue}	
	\begin{beamercolorbox}[sep=0.5em,wd=10.5cm]{postit}
		We are no longer trying to replicate the expert's behaviour!
	\end{beamercolorbox}
	\uncover<2->{$\Rightarrow$ This provides a much more clear motivation for IRL!}
	\vspace{9mm}
	\uncover<3->{
		\setbeamercolor{postit}{fg=white,bg=MyBlue}	
	\begin{beamercolorbox}[sep=0.5em,wd=7.5cm]{postit}
		As posed, the problem seems unsolvable.
	\end{beamercolorbox}}\\
	\uncover<4->{$\Rightarrow$ We need to make extra assumptions and approximations!}
	\end{centering}
\end{frame}

\/*


% PROJECT PARTNERS AND WORKPACKAGES 
\begin{frame}{Partners and Work-Packages}
	\begin{columns}
		\begin{column}[t]{2cm}
		\includegraphics[scale = 0.1]{UVA.png}
		\end{column}
		\begin{column}[t]{2cm}
		\includegraphics[scale = 0.2]{GIRAFF.jpg}
		\end{column}
		\begin{column}[t]{2cm}
		\includegraphics[scale = 0.8]{UPO.png}
		\end{column}
		\begin{column}[t]{2cm}
		\includegraphics[scale = 0.3]{IMP.jpg}
		\end{column}
		\begin{column}[t]{2cm}
		\includegraphics[scale = 0.3]{TWENTE.jpg}
		\end{column}
	\end{columns}
			\begin{itemize}
				\item \textbf{WP1} : Project Coordination and Management - UvA.
				\item \textbf{WP2} : Detecting and Undestanding Social Behaviour - Imperial, University of Twente.
				\item \textbf{WP3} : Socially normative Human-Robot Interaction - University of Twente.
				\item \textbf{WP4} : Social Navigation - Universidad Pablo de Olavide (UPO)
				\item \textbf{WP5} : Learning Social Skills - UvA.
				\item \textbf{WP6} : Integration, deployment, evaluation - Giraff.
			\end{itemize}
\end{frame}
\begin{frame}{Example}

\end{frame}

\subsection[Kurzversion des ersten Unterabschnittstitels]
{Erster Unterabschnittstitel}

\begin{frame}{�berschriften m�ssen informativ sein.\\
    Korrekte Gro�-/Kleinschreibung beachten.}{Untertitel sind optional.}
  % - Eine �berschrift fasst einen Rahmen verst�ndlich zusammen. Man
  %   muss sie verstehen k�nnen, selbst wenn man nicht den Rest des
  %   Rahmens versteht.

  \begin{itemize}
  \item
    Viel \texttt{itemize} benutzen.
  \item
    Sehr kurze S�tze oder Satzglieder verwenden.
  \end{itemize}
\end{frame}

\begin{frame}{�berschriften m�ssen informativ sein.}

  Man kann Overlays erzeugen\dots
  \begin{itemize}
  \item mit dem \texttt{pause}-Befehl:
    \begin{itemize}
    \item
      Erster Punkt.
      \pause
    \item    
      Zweiter Punkt.
    \end{itemize}
  \item
    mittels Overlay-Spezifikationen:
    \begin{itemize}
    \item<3->
      Erster Punkt.
    \item<4->
      Zweiter Punkt.
    \end{itemize}
  \item
    mit dem allgemeinen \texttt{uncover}-Befehl:
    \begin{itemize}
      \uncover<5->{\item
        Erster Punkt.}
      \uncover<6->{\item
        Zweiter Punkt.}
    \end{itemize}
  \end{itemize}
\end{frame}


\subsection{Zweiter Unterabschnittstitel}

\begin{frame}{�berschriften m�ssen informativ sein.}
\end{frame}

\begin{frame}{�berschriften m�ssen informativ sein.}
\end{frame}

\section*{Zusammenfassung}

\begin{frame}{Zusammenfassung}

  % Die Zusammenfassung sollte sehr kurz sein.
  \begin{itemize}
  \item
    Die \alert{erste Hauptbotschaft} des Vortrags in ein bis zwei Zeilen.
  \item
    Die \alert{zweite Hauptbotschaft} des Vortrags in ein bis zwei Zeilen.
  \item
    Eventuell noch eine \alert{dritte Botschaft}, aber nicht noch mehr.
  \end{itemize}
  
  % Der folgende Ausblick ist optional.
  \vskip0pt plus.5fill
  \begin{itemize}
  \item
    Ausblick
    \begin{itemize}
    \item
      Etwas, was wir noch nicht l�sen konnten.
    \item
      Nochwas, das wir noch nicht l�sen konnten.
    \end{itemize}
  \end{itemize}
\end{frame}
*/

\end{document}


